%%%%%%%%%%%%%%%%%%%%%%%%%%%%%%%%%%%%%%%%%%%%%%%%%%%%%%%%%%%%%%%%%%%%%%
% How to use writeLaTeX: 
%
% You edit the source code here on the left, and the preview on the
% right shows you the result within a few seconds.
%
% Bookmark this page and share the URL with your co-authors. They can
% edit at the same time!
%
% You can upload figures, bibliographies, custom classes and
% styles using the files menu.
%
%%%%%%%%%%%%%%%%%%%%%%%%%%%%%%%%%%%%%%%%%%%%%%%%%%%%%%%%%%%%%%%%%%%%%%

\documentclass[12pt]{article}

\usepackage{sbc-template}

\usepackage{graphicx,url}

%\usepackage[brazil]{babel}   
\usepackage[utf8]{inputenc}  

\usepackage[table]{xcolor} % Important: Load the xcolor package

\usepackage{listings}

\usepackage{fvextra}

\lstdefinestyle{mystyle}{
    language=Java,
    basicstyle=\ttfamily,
    keywordstyle=\color{blue},
    commentstyle=\color{green},
    frame=single,
    breaklines=true,
    tabsize=2, % <- This is the key line! 4 spaces per tab.
    showspaces=false, % Set to 'true' to see spaces as symbols
    showtabs=false,   % Set to 'true' to see tabs as symbols
    % Related useful option: Converts tabs to spaces in the output
    % tabreplace=true
    literate={é}{{\'e}}1
}

\lstset{style=mystyle} % Apply the style,,

     
\sloppy

\title{Exercício 1 Sistemas Distribuídos}

\author{Christian Aguiar Plentz\inst{1}, João Roberto Lemos Fidellis\inst{2} }


\address{Ciência da Computação -- Universidade do Vale do Rio dos Sinos (UNISINOS)\\
  Caixa Postal 275 -- 93022-750 -- São Leopoldo -- RS -- Brazil
  \email{\{plentzchristian,JRFIDELLIS\}@edu.unisinos.br}
}

\begin{document} 

\maketitle

\section{Exercícios} \label{sec:exercicios}

A seguir são os exercícios da aula 3 de Sistemas Distribuídos do Grau A

\subsection{Por que sistemas em rede são organizados em camadas?}

Cada camada da Internet provem utilidade para a camada acima usufruir com a abstração
da camada subsequente. Então um programador não precisa se preocupar com a responsabilidade
da integridade dos dados mandados se ele usar a tecnologia TCP/IP que já mantém essa parte
de um programa.

\subsection{Quais as vantagens da arquitetura "Computadores em Rede"?} 

A vantagem é que dependendo da implementação de responsabilidade de cada computador nessa rede
se um Computador em uma rede cai, todo o serviço não cai junto com ele.

\subsection{Por que a arquitetura cliente-servidor tem problemas de escalabilidade?}

Porque, à medida que o número de clientes cresce, o servidor central vira um ponto único de sobrecarga: ele precisa processar todas as requisições, o que pode gerar lentidão, gargalos e até falhas se não houver recursos suficientes.

\subsection{Por que replicação funciona bem quando a maioria das operações forem de leitura?} 

Porque operações de escrita tem a complexidade relacionada a propagar os dados em todas as réplicas. Se escrever em uma das réplicas, preciso no mínimo invalidar o dado nas demais réplicas ou substituir com a nova versão da informação.

\subsection{O que você entende pelo processo de reflexão (Java)?} 
O processo de reflexão consiste no processo consultando informações sobre classes e objetos em tempo de execução. Exemplo: Durante execução, chamar um método de \textbf{java.lang.reflect} para obter uma lista dos métodos de uma classe

\section{Implementacao RMI Java}\label{sec:impl_rmi}

\subsection{Código RMI, execute e comente o código e a execução. Pode ser qualquer código.}

\begin{lstlisting}[caption={Classe CalcImpl e Calc}, label=lst:calc_rmi]
package calcRmi;
public interface Calc extends Remote {
    int add(int i, int j) throws RemoteException;
}
public class CalcImpl extends UnicastRemoteObject implements Calc {
    public CalcImpl() throws RemoteException {
        super();
    }

    public int add(int i, int j) {
        return i + j;
    }
}
\end{lstlisting}

\begin{lstlisting}[caption={Cliente}, label=lst:calc_rmi_client]
package client;
public class Main {
	public static void main(String[] args) {
		try {
		      Calc objCalc = (Calc)Naming.lookup("//localhost:2004/Calc"); 
			//Busca o ip localhost na porta 2004 pelo Objeto Calc
            System.out.println("O resultado da soma é : " + objCalc.add(3, 7));
		} catch (Exception e) {
			e.printStackTrace();
		}
	}
}
\end{lstlisting}

\begin{lstlisting}[caption={Servidor}, label=lst:calc_rmi_server]
package server;
class Main {
	public static void main(String[] args){
		try {
			//Usa a porta 2004 para receber chamadas
			LocateRegistry.createRegistry(2004); 
			CalcImpl objRMI = new CalcImpl();
			//Ouve o IP localhost porta 2004 com a classe Calc
			Naming.rebind("//127.0.0.1:2004/Calc", objRMI);

		}	catch(Exception e) {
			e.printStackTrace();
		}
	}
}
\end{lstlisting}

O servidor não imprime nada mas quando o servidor esta rodando e o cliente for rodado
o cliente tera este Output no terminal.

\begin{Verbatim}
$ java client.Main 
O resultado da soma é : 10
\end{Verbatim}

Se o servidor não estiver rodando ele vai lançar uma exceção de que a conexão foi rejeitada

\begin{Verbatim}[breaklines=true, breakanywhere=true]
$ java client.Main
java.rmi.ConnectException: Connection refused to host: localhost; nested exception is: 
        java.net.ConnectException: Connection refused
        at java.rmi/sun.rmi.transport.tcp.TCPEndpoint.newSocket(TCPEndpoint.java:625)
        at java.rmi/sun.rmi.transport.tcp.TCPChannel.createConnection(TCPChannel.java:217)
        at java.rmi/sun.rmi.transport.tcp.TCPChannel.newConnection(TCPChannel.java:204)
        at java.rmi/sun.rmi.server.UnicastRef.newCall(UnicastRef.java:345)
        at java.rmi/sun.rmi.registry.RegistryImpl_Stub.lookup(RegistryImpl_Stub.java:116)
        at java.rmi/java.rmi.Naming.lookup(Naming.java:101)
        at client.Main.main(Main.java:9)
Caused by: java.net.ConnectException: Connection refused
        at java.base/sun.nio.ch.Net.connect0(Native Method)
        at java.base/sun.nio.ch.Net.connect(Net.java:589)
        at java.base/sun.nio.ch.Net.connect(Net.java:578)
        at java.base/sun.nio.ch.NioSocketImpl.connect(NioSocketImpl.java:583)
        at java.base/java.net.SocksSocketImpl.connect(SocksSocketImpl.java:327)
        at java.base/java.net.Socket.connect(Socket.java:751)
        at java.base/java.net.Socket.connect(Socket.java:686)
        at java.base/java.net.Socket.<init>(Socket.java:555)
        at java.base/java.net.Socket.<init>(Socket.java:324)
        at java.rmi/sun.rmi.transport.tcp.TCPDirectSocketFactory.createSocket(TCPDirectSocketFactory.java:40)
        at java.rmi/sun.rmi.transport.tcp.TCPEndpoint.newSocket(TCPEndpoint.java:619)
        ... 6 more
\end{Verbatim}


\end{document}

